\section{Introduction and Efforts Allocation}

\subsection{Introduction}

In today's society, fostering social responsibility and sustainable habits has become a fundamental challenge. Despite increasing awareness of environmental and social issues, many people still lack consistent motivation to engage in positive daily actions. The goal of this project is to create a digital tool that transforms good deeds into a rewarding and collaborative experience.

The project \textbf{Karma} is a mobile application designed to promote positive social and environmental behavior among users. Its main concept is to encourage individuals to perform beneficial actions (such as sustainable or socially valuable activities) and reward them with \textit{Karma Points}. These points can later be used to publish new actions for others to complete, thereby creating a collaborative and virtuous cycle of good deeds.

In addition to user-driven actions, the app also integrates sustainability-oriented tasks such as properly disposing of waste. Using open data sources, the application will display nearby recycling or garbage collection points, helping users act responsibly and conveniently. Furthermore, Karma will incorporate passive actions like step tracking, allowing users to gain points for maintaining healthy daily routines.

Accessibility is a key component of the design. The application will be optimized for users with visual impairments or limited visibility, ensuring that it can be used effectively by everyone. Also, Karma aligns with the following \textbf{\glspl{SDG}} \cite{17GOALSSustainable}:
\begin{itemize}
    \item \textbf{\gls{SDG} 3: Good Health and Well-being}, by promoting physical activity.
    \item \textbf{\gls{SDG} 11: Sustainable Cities and Communities}, through responsible waste management actions.
    \item \textbf{\gls{SDG} 12: Responsible Consumption and Production}, by encouraging individuals to adopt sustainable habits in their everyday lives.
    \item \textbf{\gls{SDG} 13: Climate Action}, by fostering environmentally friendly behavior and awareness.
\end{itemize}

Beyond its functional scope, Karma is conceived as a gamified ecosystem that combines technology, community engagement, and sustainability. By transforming everyday actions into challenges and achievements, the app leverages the motivational power of game design to foster long-term behavioral change.

\subsection{Efforts Allocation}

The development of the project will be carried out collaboratively as follows:
\begin{itemize}
    \item \textbf{Member 1:} User Interface and accessibility implementation.
    \item \textbf{Member 2:} Data integration, including the use of open data sources and \acrshort{MQTT}.
    \item \textbf{Member 3:} Sensor management, user profile, and leaderboard and chart visualization.
\end{itemize}

All members will contribute to testing, documentation, and final presentation.
