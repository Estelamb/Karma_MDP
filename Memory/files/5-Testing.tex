\section{System Testing and Validation}

The \textbf{Karma} application underwent a structured system testing and validation process to ensure it met the functional, accessibility, and usability requirements defined during the design and development phases. This process was essential to guarantee that the app delivers a seamless and user-friendly experience.

\subsection{Functional Testing}

Functional testing focused on verifying that all core features of the application worked as intended across multiple Android devices. The main modules, including mission creation and completion, Karma Points calculation, leaderboard updates, step counter tracking, and profile management, were thoroughly tested. 

All functionalities operated correctly, demonstrating the application's reliability and consistency under normal usage conditions. This ensured that the system behaves as expected and effectively supports the user in completing tasks, tracking progress, and engaging with the app's core features.

\subsection{Accessibility Testing}

Accessibility testing was conducted using the \textbf{Accessibility Scanner} \cite{GetStartedAccessibility}, which helped identify potential issues related to text size, contrast, labeling, and interactive elements. Adjustments were made to improve the readability of key content, the clarity of navigational elements, and the touch target sizes for buttons and other interactive components. 

Through this process, it was confirmed that \textbf{Karma} is usable by individuals with varying visual abilities and meets common accessibility guidelines. Ensuring accessibility early in development helped make the app more inclusive, allowing all users to fully benefit from its features.

\subsection{User Testing}

User testing involved three participants with different backgrounds to evaluate usability and overall experience. This step was particularly valuable for identifying subtle usability issues and validating that the app is engaging for a diverse user base:

\begin{itemize}
    \item \textbf{Elderly user:} Interacted primarily with the main screen, missions navigation, and step counter to assess ease of use and clarity of information.
    \item \textbf{Peer user:} Explored missions, the leaderboard, and the profile screen to evaluate engagement, understanding of features, and motivational elements.
    \item \textbf{Student user:} Tested all functionalities of the application, focusing on navigation, visual design, and the gamification elements such as Karma Points accumulation and achievement tracking.
\end{itemize}

This combination of functional, accessibility, and user testing provided comprehensive validation, confirming that the application delivers a reliable, inclusive, and engaging experience aligned with its design goals. These testing efforts ensured that \textbf{Karma} is both practical and enjoyable for all users, fulfilling its intended purpose effectively.
